\chapter{Anspruch des K gegen W auf Übergabe von 50FL Weißburgunder aus § 433 I}\label{chap:433}
\section{Anspruch}
Besteht zwischen K und W ein gültiger Kaufvertrag nach § 433 BGB,
hat K einen Anspruch auf die Übergabe der gekauften Sache durch W.\footnote{Vgl. \cite[§433, Rn.1)]{bgb17}.}
\TODO{§433 I Gerichtsurteil + Kommentar}


\section{Vertretung}
Da W in dieser Angelegenheit nicht unmittelbar beteiligt war, kann nur dann ein gültiger Vertrag zwischen K und W entstanden sein, wenn L eine Vertretungsmacht gehabt hat.
Die Erteilung dieser Vertretungsmacht ist hier durch die mündliche Aufforderung von W an L, ihn während seiner Abewesenheit zu vertreten, laut § 167 I BGB passiert. 
Nach § 164 I BGB ist eine Willenserklärung, die mit zustehender Vertretungsmacht\TODO{§ 167 I Gerichtsurteil + Kommentar} abgegeben wird, unmittelbar gültig.
Folglich stellt die Vertretung durch L kein Wirksamkeitshindernis des Vertrags dar.

\section{Kaufvertrag}
Ein Angebot, so wie dessen Annahme, sind einseitige, empfangsbedürftige Willenserklärungen.
Durch zwei übereinstimmende Willenserklärung kommt nach § 145 BGB ein bindender Kaufvertrag zustande.
In diesem mündlichen Vertrag wurde vereinbart, dass K im Juli 2020, 50 Flaschen Weißburgunder im Tausch für 500€ erhalten soll.
Diese Pflicht wurde von W nicht eingehalten, da die Flaschen anderweitig veräußert wurden.

\section{Unmöglichkeit}
Nichterbringbarkeit der Leistung
Der Leistungsanspruch von K an W könnte nach § 275 ausgeschlossen sein.

\subsection{Schuld}
Um zu prüfen, ob in diesem Sachverhalt die Einhaltung der Pflicht von W an K im Sinne von § 275 I BGB unmöglich ist, muss festgestellt sein, um welche Art der Schuld es sich handelt.


\subsubsection{Stückschuld}
Sofern ein Gegenstand in einer Weise konkretesiert ist, sodass er nach bestimmten und individuellen Kriterien bestimmt werden kann, liegt eine Stückschuld vor.\footnote{Vgl. \cite[§243, Rn.4)]{bgb19}.}

\subsubsection{Gattungsschuld nach § 243 I BGB}
(generalisierend) In diesem Sachverhalt geht es um (allgemeinen) Weißburgunder, welcher nicht von spezieller Beschaffenheit ist.
In § 243 I wird festgelegt, dass eine Sache ...
\TODO{§ 243 I Gerichtsurteil + Kommentar}

\subsubsection{Erbringung der Schuld nach § 243 II BGB}
Also sind alle Voraussetzungen des § 275 II erfüllt. Der Anspruch des K aus § 433 I 1 ist
folglich ausgeschlossen

\subsection{Beschaffenheit}
Nach § 275 BGB wird in drei Tatbestände differenziert.\footnote{Vgl. \cite[§275, Rn.4)]{bgb20}.}


\subsubsection{Echte Unmöglichkeit}
unmölglich -- Da Ware nicht zerstört, noch möglich. Falls schon getrunken, nichtmehr möglich.

\subsubsection{Praktische Unmöglichkeit}
objektiv -- Da Ware bei Dritter Partei, kann theoretisch zurückgekauft werden. Falls Preis exoribant hoch, sich der (Preis)-Aufwand also nicht lohnt, dann unmöglich.
Sache gehört einem Dritten ... \footnote{Vgl. \cite[§433, Rn.6)]{bgb17}.}

\subsubsection{Persönliche Unmöglichkeit}
subjektiv -- aus persönlichen ansichten z.B. ... hier nicht zutreffend


\section{Gebundenheit nach § 145 BGB}
Ein Angebot ist stets bindend. \TODO{einseitige Annahmeerklärung} ...
Sonderpreis von 10€ bleibt??
\TODO{todo: 14€ bei anderem weingut, 12€ / 10€ DIfferenz, schadensersatz??}
