\chapter{Anspruch des K gegen W auf Übergabe der 50FL Weißburgunder aus § 433 I}\label{chap:433}
\section{Anspruch}
\TODO{todo: text, stichpunkte}
Besteht zwischen K und W ein gültiger Kaufvertrag nach § 433,
hat K einen Anspruch auf die Übergabe der gekauften Sache durch W.

\section{Vertretung}
Da W in dieser Angelegenheit nicht unmittelbar beteiligt war, kann nur dann ein gültiger Vertrag zwischen K und W entstanden sein, wenn L eine Vertretungsmacht gehabt hat.
Die Erteilung dieser Vertretungsmacht ist hier durch die mündliche Aufforderung von W an L, ihn während seiner Abewesenheit zu vertreten, laut § 167 I passiert. 
Nach § 164 I ist eine Willenserklärung, die mit zustehender Vertretungsmacht\TODO{fallbeispiel vertretungsmacht, bgb urteil} abgegeben wird, unmittelbar gültig.
Folglich stellt die Vertretung durch L kein Wirksamkeitshindernis des Vertrags dar.

\section{Kaufvertrag}
Ein Angebot, so wie dessen Annahme, sind einseitige, empfangsbedürftige Willenserklärungen.
Durch zwei übereinstimmende Willenserklärung kommt nach § 145 ein bindender Kaufvertrag zustande.
In diesem mündlichen Vertrag wurde vereinbart, dass K im Juli 2020, 50 Flaschen Weißburgunder im Tausch für 500€ erhalten soll.
Diese Pflicht wurde von W nicht eingehalten, da die Flaschen anderweitig veräußert wurden.

\section{Unmöglichkeit}
Nichterbringbarkeit der Leistung

\subsection{Beschaffenheit}
In § 275 wird in drei

\subsubsection{Echte Unmöglichkeit}
unmölglich
\subsubsection{Praktische Unmöglichkeit}
obj 
\subsubsection{Persönliche Unmöglichkeit}
sub

\subsection{Schuld}
Um zu prüfen, ob eine Schuld unmöglich zu erbringen ist, musst festgestellt sein, um welche Art der Schuld es sich handelt.

\subsubsection{Stückschuld}
konkret

\subsubsection{Gattungsschuld}
generalisierend

\subsubsection{Erbringung der Schuld}
Da
