\chapter{Anspruch des K gegen W auf Übereignung von 50FL Weißburgunder aus § 433 I}\label{chap:433}
\section{Anspruch}
K könnte einen Anspruch auf die Übergabe der gekauften Sache durch W nach § 433 I BGB haben.\footnote{Vgl. \cite[§433, Rn.10]{bgb17}.}
Dafür muss ein wirksamer Kaufvertrag zwischen K und W gelten.\footnote{Vgl. \cite[§433, Rn.2]{bgb17}.}
\TODO{§433 I Urteil}

\subsection{Vertretung}
Im Sinne von § 164 BGB kann eine wirksame Vertretung nur vorliegen, wenn die Vertretende Person eine Vertretungsmacht im Namen des Vertretenden erhalten hat.
Da W in dieser Angelegenheit nicht unmittelbar beteiligt war, kann folglich nur dann ein gültiger Vertrag zwischen K und W entstanden sein, wenn L eine Vertretungsmacht gehabt hat.
Die Erteilung dieser Vertretungsmacht ist hier durch die mündliche Aufforderung von W an L, ihn während seiner Abwesenheit zu vertreten, laut § 167 I BGB passiert.
Auf Grund dieser Vertretungsmacht ist L berechtigt im Namen von W, Willenserklärungen nicht nur abzugeben, sondern auch entgegen zu nehmen.
Nach § 164 I BGB ist eine Willenserklärung, die mit zustehender Vertretungsmacht abgegeben wird, unmittelbar gültig.
Folglich stellt die Vertretung durch L kein Wirksamkeitshindernis des Vertrags dar.
K kann folglich mit L im Sinne von W ein Rechtsgeschäft abschließen.
\TODO{§ 167 I Urteil + Kommentar}

\subsection{Kaufvertrag}
Damit ein Anspruch erhoben werden kann, muss ein gültiger Vertrag zwischen K und W vorliegen.
Ein Angebot, so wie dessen Annahme, sind einseitige, empfangsbedürftige Willenserklärungen.
Durch zwei übereinstimmende Willenserklärung kommt nach § 145 BGB ein bindender Kaufvertrag zustande.
In dem mündlichen Vertrag zwischen K und W wurde vereinbart, dass K im Juli 2020, das Eigentum von 50 Flaschen Weißburgunder gemäß § 929 BGB übertragen wird.
Diese aus § 433 I BGB entstehende Pflicht wurde von W nicht eingehalten, da die Flaschen anderweitig veräußert wurden.
Ein Schuldverhältniss zwischen Käufer und Verkäufer, namentlich K und W, liegt dementsprechend vor.


\section{Unmöglichkeit}
Der Leistungsanspruch von K gegen W könnte nach § 275 ausgeschlossen sein.
Dazu muss nach §§ 275 I, II, III BGB eine Unmöglichkeit vorliegen.
Eine Unmöglichkeit ist die permanente Nichterbringbarkeit der Leistung durch den Schuldner.


\subsection{Schuld}
Um zu prüfen, ob in diesem Sachverhalt die Einhaltung der Pflicht von W an K im Sinne von § 275 BGB unmöglich ist, muss festgestellt sein, um welche Art der Schuld es sich handelt.\footnote{Vgl. \cite[§275, Rn.4]{bgb20}.}
Die Art der Schuld ist insofern relevant, da auf Grund der Beschaffenheit der Schuld, die Kriterien der Unmöglichkeit definiert werden.
Dementsprechend wird vorausgesetzt die Art der Schuld zu kennen, bevor auf Unmöglichkeit geprüft werden kann.
\TODO{275 Urteil}


\subsubsection{Stückschuld}
Sofern ein Gegenstand in einer Weise konkretisiert ist, sodass er nach bestimmten und individuellen Kriterien bestimmt werden kann, liegt eine Stückschuld vor.\footnote{Vgl. \cite[§243, Rn.4]{bgb19}.}
\TODO{warum ist das wichtig}


\subsubsection{Gattungsschuld nach § 243 I BGB}
In diesem Sachverhalt geht es um Weißburgunder allgemeiner Gattung, welcher nicht von spezieller Beschaffenheit ist.
Dementsprechend gibt es mehrere erfüllungstaugliche Gegenstände.\footnote{Vgl. \cite[§243, Rn.3]{bgb20}.}
In § 243 I wird festgelegt, dass die Schuld einer Sache, welche nur nach der Gattung bestimmt ist, durch eine Sache mittlerer Art und Güte zu leisten ist.
\TODO{§ 243 I Urteil}


\subsubsection{Konkretisierung nach § 243 II BGB}

Die Gattungsschuld könnte nach § 243 II zu einer Stückschuld konkretisiert werden, wodurch ebenfalls eine Unmöglichkeit in Sinne von § 275 BGB eintritt.\footnote{Vgl. \cite[§243, Rn.25]{mueko}}
Dafür muss der Schuldner alles seinerseits Erforderliche getan haben, um den Leistungserfolg herbeizuführen.
Um zu prüfen, ob der Schuldner das seinerseits Erforderliche getan hat, wird im Rahmen der Konkretisierung zwischen Hol-, Schick- und Bringschuld unterschieden.
Aus den daraus resultierenden Voraussetzungen kann eine Nichterbringbarkeit der Leistung durch den Schuldner entstehen.

\subthreesection{Holschuld}
Bei einer Holschuld wird der Gegenstand vom Gläubiger beim Schuldner abgeholt.
Der Ort des Leistung, sowie der des Erfolges liegen beim Schuldner.\footnote{Vgl. \cite[§243, Rn.7]{bgb19}.}
Nach §243 II BGB tritt bei einer Holschuld die Konkretisierung von einer Gattungsschuld zu einer Stückschuld ein,
sobald eine bestimmte Sache dieser Gattung ausgesondert und bereitgestellt wird, sowie der Gläubiger aufgefordert, diese Sache abzuholen.
Hat der Schuldner die genannten Voraussetzungen erfüllt, hat er alles seinerseits erforderliche, im Rahmen einer Holschuld getan.

\subthreesection{Schickschuld}
Die Schickschuld setzt voraus, dass der Schuldner sich verpflichtet, die Ware zu versenden.
Dementsprechend ist der Ort der Leistung beim Schuldner und der Erfolgsort ist der Wohnsitz des Gläubigers.
Anders als bei der Holschuld, tritt die Konkretisierung hier nach der Übergabe an eine Transportperson, wie beispielsweise die Post, ein.
Sobald der Schuldner die Ware übergeben hat, hat er alles seinerseits erforderliche getan und die Gattungsschuld wird zu einer Stückschuld konkretisiert. 

\subthreesection{Bringschuld}
Bei der Bringschuld wird, anders als bei der Schickschuld, die Ware persönlich durch den Schuldner übergeben.
Leistungs- und Erfolgsort ist demnach der Wohnsitz des Gläubigers.
Hier tritt die erst Konkretisierung ein, sobald die Ware tatsächlich am Wohnsitz des Gläubigers angeboten wird.

\subthreesection{Konkretisierung im Rechtsgeschäft von K und W}
In dem Rechtsgeschäft zwischen K und W befindet sich der Leistungs- und Erfolgsort beim Wohnsitz, gemäß § 7 BGB, des Schuldners.
Folglich liegt zwischen K und W eine Holschuld vor.
Die erforderlichen Pflichten des Schuldners im Rahmen einer Holschuld belaufen sich auf das Bereitstellen der Ware, sowie die Aufforderung an den Gläubiger, diese abzuholen.
Das Angebot von W an K über den Weißwein richtet sich konkret an 50 Flaschen, welche für K vorgesehen waren.
W stellt die Aufforderung an K, den Wein im Juli abzuholen.
Dementsprechend sind die Voraussetzungen zur Konkretisierung einer Gattungschuld nach § 243 II BGB erfüllt und die eigentliche Gattungsschuld wird wie eine Stückschuld behandelt.


\subsection{Beschaffenheit}
Nach §§ 275 I, II, III BGB wird in drei verschiedene Tatbestände, welche zu der Unmöglichkeit führen, differenziert.
Die Unterscheidung wird anhand der zur Unmöglichkeit führenden Ursachen bestimmt. 


\subsubsection{Echte Unmöglichkeit}
In § 275 I BGB wird die echte Unmöglichkeit beschrieben.
Sofern die Leistung aus tatsächlicher oder rechtlicher Sicht nicht zu erfüllen ist, liegt eine echte Unmöglichkeit vor.\footnote{Vgl. \cite[§275, Rn.2]{bgb19}.}
Ob es sich um eine objektive oder subjektive Unmöglichkeit handelt, spielt rechtlich keine Rolle.
Gemäß § 275 I BGB erlischt die Leistungspflicht des Schuldners kraft Gesetzes und der Gläubiger hat keinen Anspruch auf Leistungserfüllung mehr.
In Relation zum Eintritt der Unmöglichkeit wird zwischen anfänglicher und nachträglicher Unmöglichkeit unterschieden.\footnote{Vgl. \cite[§251, Rn.7]{mueko}.}
Obwohl der Wein von W an S übereignet und anschließend von S vollständig veräußert wurde, besteht theoretisch immernoch die Gelegenheit diesen zurückzuerwerben.
Die eventuelle Unmöglichkeit wurde ausgelöst nachdem das Schuldverständnisses entstanden war.
Zum Zeitpunkt der Willenserklärung waren die Flaschen verfügbar.
Dementsprechend würde eine nachträgliche Unmöglichkeit vorliegen.\footnote{Vgl. \cite[Rn.376]{schmidt}.}
Sofern die Flaschen Wein nach der Veräußerung durch S nicht zerstört, bzw. konsumiert wurden, liegt keine tatsächliche Unmöglichkeit nach § 275 BGB I vor.
\TODO{Urteil}


\subsubsection{Faktische Unmöglichkeit}
In § 275 II BGB wird die faktische Unmöglichkeit beschrieben.
Die Leistung der Schuld durch den Schuldner ist theoretisch möglich, die Erfüllung dieser steht jedoch in einem groben Missverhältnis zu dem Leistungsinteresse des Gläubigers.
Sofern die Möglichkeit für den Schuldner besteht, die Schuld zu begleichen, der dafür nötige Aufwand diesem jedoch nicht zuzumuten ist, liegt eine faktische Unmöglichkeit vor.\footnote{Vgl. \cite[§433, Rn.6]{bgb17}.}
Anders als in § 275 I BGB, muss in § 275 II BGB das Leistungsverweigerungsrecht vom Schuldner geltend gemacht werden.
W veräußerte den Wein an S, welcher diesen vollständig weiterverkauft hat.
Da zwischen W und S gültiger Vertrag nach § 145 zu Stande gekommen ist, besteht laut § 435 BGB kein Fall der Rechtsmängelgewährleistung.\footnote{Vgl. \cite[§435, Rn.3]{bgb19}.}
Der Kaufvertrag kann dementsprechend nicht angefochten un die Ware nicht zurückverlangt werden.
Sofern kein gutgläubiger Erwerb in Frage kommt und die Dritte Partei zur Übertragung ihres Eigentums auf W oder K nicht bereit ist, liegt ein Fall der Unmöglichkeit vor.
Da durch die vergangene Hochzeit kein Interesse mehr von K am Erhalt des Weins besteht, steht somit das Leistungsinteresse des Gläubigers in keinem Verhältnis mehr zu dem aufzubringenden Aufwand des Schuldners.
Dementsprechend könnte eine Unmöglichkeit nach § 275 II BGB vorliegen. 
Ob tatsächlich ein grobes Missverhältnis zwischen dem Aufwand des Schuldners und dem Leistungsinteresse des Gläubigers besteht, muss, falls benötigt, entsprechend weiter geprüft werden.\footnote{Vgl. \cite[§275, Rn.60]{bgb20}.}
Im Rahmen dieses Gutachtens jedoch, ist der vorliegende Tatbestand Beweis genug, für eine vorliegende faktische Unmöglichkeit nach § 275 II BGB.
\TODO{Urteil}


\subsubsection{Persönliche Unmöglichkeit}
In § 275 III BGB wird die persönliche Unmöglichkeit beschrieben.
Die vorausgesetzte Unzumutbarkeit entsteht durch nicht wirtschaftliche Gründen.
Nur Leistungen, welche persönlich erbracht werden müssen, werden durch § 275 III BGB erfasst.
Das Leistungshinderniss stellt die Person des Schuldners dar.
So werden persönliche Umstände, sowie moralische Gründe berücksichtigt.\footnote{Vgl. \cite[§275, Rn.23]{bgb19}.}
Die Erbrachte Schuld von W an K muss nicht persönlich erbracht werden, daher kann gemäß § 275 III BGB keine Unmöglichkeit eintreten.


\subsection{Rechtsfolge}
Wenn die Unmöglichkeit bejaht wird bleibt der Vertrag wirksam.\footnote{Vgl. \cite[§275, Rn.24]{bgb19}.}
Die Wirksamkeit des Vertrags bleibt unberührt, da sich die Rechtfolgen der Unmöglichkeit nur auf die Abwicklung beziehen.
Anders als bei § 275 I BGB, erlischt die Leistungspflicht des Schuldners aus § 275 II BGB nicht durch das Gesetz, sondern nur wenn Einspruch seinerseits erhoben wird.
Das Erbringen der Pflicht ist auf Grund des Leistungsverweigerungsrecht des Schuldners nichtmehr erforderlich.
Die Leistung des Schuldners wird nicht erbracht.
Ein Anspruch auf Übergabe des Weins nach §§ 433 I, 929 BGB besteht von K gegen W nicht.
Der Gläubiger kann dann mögliche Schäden, welche durch das Fehlen der eigentlichen Leistung zu Stande gekommen sind, begleichen, indem er gemäß § 280 I BGB Anspruch auf Ersatz des Schadens durch den Schuldner erhebt.
\TODO{Urteil + Kommentar}
