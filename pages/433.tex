\chapter{Anspruch des K gegen W auf Übergabe der 50FL Weißburgunder aus § 433 I}\label{chap:433}
\TODO{todo: text, stichpunkte}
\section{Anspruch}
Besteht zwischen K und W ein gültiger Kaufvertrag nach § 433,
könnte K einen Anspruch auf die Übergabe der gekauften Sache durch W haben.

\section{Vertretung}
Ein gültiger Vertrag zwischen K und W kann nur vorliegen, wenn L eine Vertretungsmacht gehabt hat.
Dieses ist hier durch die mündliche Aufforderung von W an L, ihn während seiner Abewesenheit zu vertreten, laut § 167 I passiert. 
Nach § 164 I ist eine Willenserklärung, die mit zustehender Vertretungsmacht abgegeben wird, unmittelbar gültig.
Folglich stellt die Vertretung durch L kein Wirksamkeitshindernis dar.

\section{Kaufvertrag}
Ein Angebot, so wie dessen Annahme, sind einseitige, empfangsbedürftige Willenserklärungen.
Durch zwei übereinstimmende Willenserklärung kommt nach § 145 ein bindender Kaufvertrag zustande.




\subsection{test}
