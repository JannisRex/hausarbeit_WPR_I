\chapter{Anspruch des K gegen W auf Übergabe von 50FL Weißburgunder aus § 433 I}\label{chap:433}
\section{Anspruch}
K könnte einen Anspruch auf die Übergabe der gekauften Sache durch W nach § 433 I BGB haben.\footnote{Vgl. \cite[§433, Rn.10)]{bgb17}.}
Dafür muss ein wirksamer Kaufvertrag zwischen K und W gelten.\footnote{Vgl. \cite[§433, Rn.2)]{bgb17}.}
\TODO{§433 I Urteil}


\subsection{Vertretung}
Da W in dieser Angelegenheit nicht unmittelbar beteiligt war, kann nur dann ein gültiger Vertrag zwischen K und W entstanden sein, wenn L eine Vertretungsmacht gehabt hat.
Die Erteilung dieser Vertretungsmacht ist hier durch die mündliche Aufforderung von W an L, ihn während seiner Abewesenheit zu vertreten, laut § 167 I BGB passiert. 
Nach § 164 I BGB ist eine Willenserklärung, die mit zustehender Vertretungsmacht abgegeben wird, unmittelbar gültig.
Folglich stellt die Vertretung durch L kein Wirksamkeitshindernis des Vertrags dar.
\TODO{§ 167 I Urteil + Kommentar}

\subsection{Kaufvertrag}
Ein Angebot, so wie dessen Annahme, sind einseitige, empfangsbedürftige Willenserklärungen.
Durch zwei übereinstimmende Willenserklärung kommt nach § 145 BGB ein bindender Kaufvertrag zustande.
In diesem mündlichen Vertrag wurde vereinbart, dass K im Juli 2020, das Eigentum von 50 Flaschen Weißburgunder gemäß § 929 BGB erwirbt.
Diese Pflicht wurde von W nicht eingehalten, da die Flaschen anderweitig veräußert wurden.
Ein Schuldverhältniss liegt dementsprechend vor.


\section{Unmöglichkeit}
Der Leistungsanspruch von K an W könnte nach § 275 ausgeschlossen sein.
Unmöglichkeit ist die Nichterbringbarkeit der Leistung durch den Schuldner.
\TODO{Text}

\subsection{Schuld}
Um zu prüfen, ob in diesem Sachverhalt die Einhaltung der Pflicht von W an K im Sinne von § 275 BGB unmöglich ist, muss festgestellt sein, um welche Art der Schuld es sich handelt.\footnote{Vgl. \cite[§275, Rn.4)]{bgb20}.}
\TODO{275 Urteil}
\TODO{Text, " das ist insofern relevant da...."}


\subsubsection{Stückschuld}
Sofern ein Gegenstand in einer Weise konkretesiert ist, sodass er nach bestimmten und individuellen Kriterien bestimmt werden kann, liegt eine Stückschuld vor.\footnote{Vgl. \cite[§243, Rn.4)]{bgb19}.}
\TODO{warum ist das wichtig}


\subsubsection{Gattungsschuld nach § 243 I BGB}
In diesem Sachverhalt geht es um Weißburgunder allgemeiner Gattung, welcher nicht von spezieller Beschaffenheit ist.
Dementsprechend gibt es mehrere erfüllungstaugliche Gegenstände. \footnote{Vgl. \cite[§243, Rn.3]{bgb20}.}
In § 243 I wird festgelegt, dass die Schuld einer Sache, welche nur nach der Gattung bestimmt ist, durch eine Sache mittlerer Art und Güte zu leisten ist.
\TODO{§ 243 I Urteil}


\subsubsection{Konkretisierung nach § 243 II BGB}

Die Gattungsschuld könnte nach § 243 II zu einer Stückschuld konkretisiert werden, wodurch ebenfalls eine Unmöglichkeit eintritt.\footnote{Vgl. \cite[§243, Rn.25]{mueko}}
Dafür muss der Schuldner alles seinerseits Erforderliche getan haben, um den Leistungserfolg herbeizuführen.
Um zu prüfen, ob der Schuldner das seinerseits Erforderliche getan hat, muss festgestellt werden, um welche Art der Schuld es sich handelt.
Bei einer Holschuld wird der Gegenstand vom Gläubiger beim Schuldner abgeholt.
Der Ort des Leistung, sowie der des Erfolges sind beim Schuldner.\footnote{Vgl. \cite[§243, Rn.7]{bgb19}.}
Nach §243 II BGB tritt bei einer Holschuld die Konkretisierung von einer Gattungsschuld zu einer Stückschuld ein,
sobald eine bestimmte Sache dieser Gattung ausgesondert und bereitgestellt wird, sowie der Gläubiger aufgefordert, diese Sache abzuholen.
Das Angebot von W an K über den Weißwein richtet sich konkret an 50 Flaschen, welche für K vorgesehen waren.
Folglich sind die Voraussetzungen erfüllt und die eigentliche Gattungsschuld wird wie eine Stückschuld behandelt.



\subsection{Beschaffenheit}
Nach § 275 BGB wird in drei Tatbestände, welche zu der Unmöglichkeit führen, differenziert.
\TODO{text}



\subsubsection{Echte Unmöglichkeit}
In § 275 I BGB wird die echte Unmöglichkeit beschrieben.
Sofern die Leistung aus tatsächlicher oder rechtlicher Sicht nicht zu erfüllen ist, liegt eine echte Unmöglichkeit vor.\footnote{Vgl. \cite[§275, Rn.2]{bgb19}.}
Ob es sich um eine objektive oder subjektive Unmöglichkeit handelt, spielt rechtlich keine Rolle.
In Relation zum Eintritt der Unmöglichkeit wird zwischen anfänglicher und nachträglicher Unmöglichkeit unterschieden.\footnote{Vgl. \cite[§251, Rn.7]{mueko}.}
Der von W an S übereignete Wein könnte theoretisch zurückerworben werden.
Die eventuelle Unmöglichkeit wurde ausgelöst nachdem das Schuldverständnisses entstanden war.
Zum Zeitpunkt der Willenserklärung waren die Flaschen verfügbar.
Dementsprechend würde eine nachträgliche Unmöglichkeit vorliegen.
Sofern die Flaschen Wein von S nicht zerstört, bzw. konsumiert wurden, liegt keine tatsächliche Unmöglichkeit nach § 275 BGB I vor.
\TODO{Urteil}


\subsubsection{Faktische Unmöglichkeit}
In § 275 II BGB wird die faktische Unmöglichkeit beschrieben.
Die Leistung der Schuld durch den Schuldner ist theoretisch möglich, die Erfüllung dieser steht jedoch in einem groben Missverhältnis zu dem Leistungsinteresse des Gläubigers.
Sofern die Möglichkeit für den Schuldner besteht, die Schuld zu begleichen, der dafür nötige Aufwand diesem jedoch nicht zuzumuten ist, liegt eine faktische Unmöglichkeit vor.\footnote{Vgl. \cite[§433, Rn.6]{bgb17}.}
Anders als in § 275 I BGB, muss in § 275 II BGB das Leistungsverweigerungsrecht vom Schuldner geltend gemacht werden.
W veräußerte den Wein an S, welcher diesen vollständig weiterverkauft hat.
Da zwischen W und S gültiger Vertrag nach § 145 zu Stande gekommen ist, besteht laut § 435 kein Fall der Rechtsmängelgewährleistung\footnote{Vgl. \cite[§435, Rn.3]{bgb19}.}
Sofern kein gutgläubiger Erwerb in Frage kommt und die Dritte Partei zur Übertragung ihres Eigentums auf W nicht bereit ist, liegt ein Fall der Unmöglichkeit vor.
Da durch die vergangene Hochzeit kein Interesse mehr von K am Erhalt des Weins besteht, könnte eine Unmöglichkeit nach § 275 II BGB vorliegen. 
Ob ein grobes Missverhältnis zwischen dem Aufwand des Schuldners und dem Leistungsinteresse des Gläubigers besteht, muss, falls benötigt, entsprechend weiter geprüft werden.\footnote{Vgl. \cite[§275, Rn.60]{bgb20}.}
\TODO{Urteil}



\subsubsection{Persönliche Unmöglichkeit}
In § 275 III BGB wird die persönliche Unmöglichkeit beschrieben.
Die vorausgesetzte Unzumutbarkeit entsteht durch nicht wirtschaftliche Gründen.
Nur Leistungen, welche persönlich erbracht werden müssen, werden durch § 275 III BGB erfasst.
Das Leistungshinderniss stellt die Person des Schuldners dar.
So werden persönliche Umstände, sowie moralische Gründe berücksichtigt.\footnote{Vgl. \cite[§275, Rn.23]{bgb19}.}
Die Erbrachte Schuld von W an K muss nicht persönlich erbracht werden, daher kann gemäß § 275 III BGB keine Unmöglichkeit eintreten.


\subsection{Rechtsfolge}
Wenn die Unmöglichkeit bejaht wird bleibt der Vertrag wirksam.\footnote{Vgl. \cite[§275, Rn.24]{bgb19}.}
Die Wirksamkeit des Vertags bleibt unberührt, da sich die Rechtfolgen der Unmöglichkeit nur auf die Abwicklung beziehen.
Anders als bei § 275 I BGB, erlischt die Leistungspflicht des Schuldners aus § 275 II BGB nicht durch das Gesetz, sondern nur wenn Einspruch seinerseits erhoben wird.
Das Erbringen der Pflicht ist auf Grund des Leistungsverweigerungsrecht des Schuldners nichtmer erforderlich.
Die Leistung wird nicht erbracht.
Der Gläubiger kann dann mögliche Schäden, welche durch das Fehlen der eigentlichen Leistung zu Stande gekommen sind, begleichen, indem er gemäß § 280 I BGB Anspruch auf Ersatz des Schadens durch den Schuldner erhebt.
\TODO{Urteil + Kommentar}
