\chapter{Anspruch des K gegen W auf Schadensersatz nach § 280 I BGB}\label{chap:280}
\section{Anspruch}
K könnte gegen W einen Anspruch auf Schadensersatz im Sinne von § 280 I BGB auf Grund von nicht Einhaltens der Pflichten des Schuldners haben.
Damit eine Pflichtverletzung vorliegen kann, muss zwischen den Partein ein Schuldverhältnis bestehen.
Gemäß § 311 I BGB wird ein Vertrag für die Begründung eines Schuldverhältnis durch Rechtsgeschäft vorausgesetzt.\footnote{Vgl. \cite[§311, Rn.1]{bgb09}.}

\subsection{Schuldverhältnis}
In einem Schuldverhältnis hat der Gläubiger gemäß § 241 I BGB das Recht eine Leistung von Schuldner zu fordnern.  
L hat im Sinne von W ein Angebot, welches die Übertragung des Eigentums von 50FL Wein an K festlegt, abgegeben.
K akzeptierte dieses Angebot, folglich ist ein gültiger Vertrag, so wie ein Schuldverhältnis zwischen K und W zu Stande gekommen.
Aus § 433 I BGB hat W die Pflicht der Übergabe der Sache, während K nach § 433 II BGB die Pflicht hat, den Kaufpreis zu zahlen, um diese Sache abzunehmen.
\TODO{Urteil + Kommentar}


\subsection{Pflichtverletzung}
Liegt durch den Schuldner eine Pflichtverletzung aus dem Schuldverhältnis vor, so hat der Gläubiger nach § 280 I BGB Anspruch auf Ersatz des entstandenen Schadens.
Eine Pflichtverletzung liegt vor, wenn die Leistungspflicht des Schuldners, oder der einer anderen Partei, nicht erfüllt wurde.
W hat seine Pflicht gemäß § 433 I BGB durch Ausüben des Leistungsverweigerungsrecht aus § 275 II BGB nicht erfüllt.
Durch den Ausfall der Leistung von W an K, muss K bei dem Nachbarweingut 50 Flaschen Wein zum Preis von 14€, statt der kalkulierten 10€ pro Flasche bezahlen.
Auf Grund der Dringlichkeit durch die anstehende Hochzeit, besteht keine Möglichkeit ein besseres Angebot zu bekommen.
Dementsprechend steht K ein Anspruch auf Schadensersatz gemäß §§ 280 I, 283 BGB zu.
\TODO{Urteil + Kommentar}


\section{Schadensersatz statt der Leistung nach §§ 280 III, 283 BGB}
K könnte gegen W Anspruch auf Schadensersatz statt der Leistung aus §§ 280 III, 283 BGB haben.
Die eigentliche Leistung kann auf Grund der nach § 275 II BGB gegebenen Unmöglichkeit nichtmehr vom Gläubiger verlangt werden kann.
Jedoch hat der Gläubiger gemäß §§ 280 III, 283 BGB einen Anspruch auf Schadensersatz.
Liegt nach §§ 275 I, II, III BGB ein Leistungsverweigerungsrecht des Schuldners vor, kann der Gläubiger unter den Voraussetzungen des § 280 I Schadensersatzansprüche gegen W stellen.
\TODO{Urteil + Kommentar}

\section{Höhe des Schadensersatz}
In § 249 BGB wird der Schadensersatz als Annäherung an den theoretischen Zustand definiert, der bestehen würde, wenn der zu entschädigende Umstand nicht eingetreten wäre.
Die Pflichtverletzung von W gegenüber K im Sinne des Leistungsausfalls nach § 433 I BGB stellt den Umstand dar.
Auf Grund dieser Pflichtverletzung musste K ein schlechteres Angebot annehmen, wodurch der Schaden entstanden ist.
Die Höhe des Schadensersatz beläuft sich auf die Differenz zwischen Preis des Angebots von W und dem gezahlten Preis von K bei dem Nachbarweingut.
Durch den verbindlich vereinbarten Sonderpreis mit W von 10€ pro Flasche hätte K nur die Gesamtsumme von 500€, statt der vom Nachbarweingut verlangten 700€, zahlen müssen.
Folglich steht K nach §§ 280 I, III, 283 BGB ein Schadensersatz in Höhe der Differenz von 200€ von W zu.
\TODO{rechnung?? oder nicht}
