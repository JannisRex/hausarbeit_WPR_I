\chapter{Anspruch des K gegen F auf x aus § 119 Anfechtbarkeit wegen Irrtums}\label{chap:119}
\TODO{todo: text, stichpunkte}
\section{Irrtum}
Liegt durch fehlende Informationen über die Wahrheit eine Abweichung der Vorstellung von der Wirklichkeit vor, so wird von Irrtum gesprochen.\footnote{Vgl. \cite[S. 302, 24 a)]{bgb17}.}
\TODO{Gerichtsurteil + Kommentar}
\section{Anfechtbarkeit}
Die Nachweisbarkeit des Irrtums steht hier außer Frage, da K bei Kentniss über die Tatsache, dass der Weißherbst ein Rosé und kein Weißwein ist, diesen nicht bestellt hätte.
Unter Kentniss der Sachlage hätte K eine solche Willenserklärung niemals abgegeben.
Unnmitelbar nach Erhalt der Lieferung informiert K F, dass er Weißwein und keinen Rosé habe kaufen wollen.
K fechtet gem. § 119 I den Kaufvertrag wegen Irrtums an.
\TODO{Rose kein Rotwein + Gerichtsurteil}