\chapter{Anspruch des K gegen F auf Anfechtung aus § 119 I BGB}\label{chap:119}

\section{Anspruch}
K könnte gegen F den Anspruch auf Anfechtung des Rechtsgeschäfts im Sinne von §§ 119 I, 143 BGB haben.
Damit der Anspruch für K besteht, muss ein rechtmäßiger Anfechtungsgrund im Sinne von § 142 BGB vorliegen.
Eine Anfechtbarkeit wird gemäß § 119 I BGB durch das Vorliegen eines Irrtums gegeben.\footnote{Vgl. \cite[§119, Rn.1]{brox}.}

\section{Irrtum}
Liegt durch fehlende Informationen über die Wahrheit, eine Abweichung der Vorstellung von der Wirklichkeit vor,
so wird von Irrtum gesprochen und ein Rechtsgeschäft is im Sinne von § 119 I BGB anfechtbar.\footnote{Vgl. \cite[§119, Rn.24]{bgb17}.} \footnote{BGH, 3.6.2006, Az. IV ZB 39/05}

\subsection{Inhaltsirrtum}
Bei einem Inhaltsirrtum bezieht sich der Umfang des Irrtums auf die Beschaffenheit des Gegenstands.\footnote{Vgl. \cite[§119, Rn.12]{bgb20}.} \footnote{LG Hanau, NJW 1979, 721}
Das subjektiv Gewollte unterscheidet sich von dem objektiv Erklärtem.\footnote{Vgl. \cite[§119, Rn.12]{palandt}.}
K wusste nicht, dass der Inhalt der Flasche kein Weißwein ist, sondern gab seine Willenserklärung in dem Glauben ab, es sei Weißwein.
Folglich liegt zwischen K und F ein Inhaltsirrtum vor.

\subsection{Kausalität}
Voraussetzung des Anfechtungsgrundes ist die Kausalität für die Abgabe der Willenserklärung.\footnote{Vgl. \cite[§119, Rn.23]{bgb17}.}
Ohne Vorliegen des Irrtums wäre die Abgabe der Willenserklärung von K und somit auch der Vertrag, nicht zu Stande gekommen.
Der Wein wurde in dem Glauben bestellt, ein Weißwein zu sein.
Folglich ist der Irrtum maßgebend für die Willenserklärung und die Kausalität ist gegeben.

\section{Anfechtbarkeit}
Die Anfechtung erfolgt durch eine empfangsbedürftige Willenserklärung gemäß § 143 I BGB.\footnote{Vgl. \cite[§123, Rn.1]{faust}.}
Die abgegebene Erklärung gegenüber dem Anfechtungsgegner ist formlos möglich.
Es muss lediglich nach §§ 133, 157 BGB deutlich werden, dass das Rechtsgeschäft vernichtet werden soll und auf welche Sachverhalte diese Anfechtung beruht.\footnote{Vgl. \cite[§123, Rn.2]{faust}.}
Dementsprechend müsste K, auf einer unverkennbaren Art und Weise, F deutlich machen, dass und aus welchem Grund, K das Rechtsgeschäft anfechten möchte.
Unmittelbar nach Erhalt der Lieferung informiert K F, dass er Weißwein und keinen Rosé habe kaufen wollen.
Die Nachweisbarkeit des Irrtums steht hier außer Frage, da K bei Kenntnis über die Tatsache, dass der Weißherbst ein Rosé und kein Weißwein ist, diesen nicht bestellt hätte.
Unter Kenntnis der Sachlage hätte K eine solche Willenserklärung niemals abgegeben.
Somit fallen Wille und Erklärung auseinander.
K handelt gemäß § 242 BGB nach Treu und Glauben und fechtet laut § 119 I BGB den Kaufvertrag wegen Irrtums an.

\section{Anfechtungsfrist}
Gemäß § 121 BGB muss die Anfechtung unverzüglich nach Kentissnahme des Anfechtungsgrundes erfolgen.\footnote{BGH, NJW 2008, 985 Rn.18}
Eine Anfechtung nach Ablauf der Frist ist nichtmehr möglich.\footnote{BGH, NJW 1971, 891}
Dieses ist hier durch K passiert, indem direkt nach Erhalt der Bestellung die Anfechtung offenkundig preisgegeben wurde.
Die Anfechtungsfrist wurde folglich im Sinne von § 121 BGB eingehalten.

\section{Rücktritt nach § 346 I BGB}
K tritt von dem Kaufvertrag laut § 349 BGB zurück, indem er dieses gegenüber F deutlich macht.
Bei der Rücktritterklärung handelt es sich um eine einseitige, empfangsbedürftige Willenserklärung.
Durch den Rücktritt tritt die sofortige Beendigung des Schuldverhältnisses ein.\footnote{Vgl. \cite[§346, Rn.8]{bgb20}.}
Im Falle des Rücktritts ist der Gegenstand, sowie dessen gewonnener Nutzen gemäß § 346 I BGB zurückzuüberlassen.\footnote{BGH, 05.07.2016, Az. XI ZR 254/15}
Statt des ehemaligen Schuldverhältnis, besteht nun ein Rückgewährschuldverhältnis zwischen K und F.
Die Pflichten aus § 433 BGB wurden von den Partein noch nicht wahrgenommen, weswegen folglich auch kein Zurücküberlassen der Gegenstände eintreten muss.
Auf Grund des rückwirkend zerstörtem Rechtsgeschäfts, stehen K und F in keinem Kaufvertrag mehr.

\section{Neues Angebot von F an K nach §145}
Durch das neue Angebot von F an K, Weißwein in vergleichbarer Qualität zu liefern, akzeptiert F die Anfechtung von K.
Gemäß § 145 BGB ist in diesem Fall kein neuer Kaufvertrag zwischen K und F zustande gekommen, da die vorrausgesetze, direkte Annahme dieses Angebots im Sinne von § 147 BGB durch K nicht erfolgt ist.
Das Angebot erlischt und es kommt kein neuer Vertrag zu Stande.

\section{Wirkung der Anfechtung § 142 BGB}
Nach § 142 I BGB werden Willenserklärungen, die für ein angefochtenes Rechtsgeschäft abgegeben wurden so behandelt, als wären sie nie abgegeben wurden.\footnote{Vgl. \cite[§142, Rn.1, 2]{bgb17}.}
Im Sinne von § 142 I BGB tritt dadurch die unmittelbare Nichtigkeit des Vertrages ein.\footnote{BGH, 05.08.1997, Az. V ZR 29/96}
Dementsprechend ist der Vertrag zwischen K und F nach § 142 I BGB nichtig, bzw. rückwirkend vernichtet.
Die Pflichten von K und F aus §§ 433 I, II BGB, sind somit erloschen und die Bindung an den Antrag gemäß § 145 BGB nichtig.