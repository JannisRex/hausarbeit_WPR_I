\chapter{Anspruch des K gegen F auf Anfechtung aus § 119 I BGB}\label{chap:119}
\TODO{todo: text, stichpunkte}
\section{Irrtum}
Liegt durch fehlende Informationen über die Wahrheit, eine Abweichung der Vorstellung von der Wirklichkeit vor,
so wird von Irrtum gesprochen und ein Rechtsgeschäft is im Sinne von § 119 I BGB anfechtbar.
\footnote{Vgl. \cite[§119, Rn.24]{bgb17}.}
\TODO{Gerichtsurteil}

\subsection{Inhaltsirrtum}
Das subjektiv Gewollte unterscheidet sicht von dem objektiv Erklärtem.
Der Irrtum umfasst die Beschaffenheit des Gegenstands.
K wusste nicht, dass der Inhalt der Flasche kein Weißwein ist.
Folglich liegt ein Inhaltsirrtum vor.

\subsection{Kausalität}
Voraussetzung des Anfechtungsgrundes ist die kausalität für die Abgabe der Willenserklärung.
\footnote{Vgl. \cite[§119, Rn.23]{bgb17}.}
Der Wein wurde in dem Glauben bestellt, ein Weißwein zu sein.
Folglich ist ist der Irrtum maßgebend für die Willenserklärung.


\section{Anfechtbarkeit}
Die Nachweisbarkeit des Irrtums steht hier außer Frage, da K bei Kentniss über die Tatsache, dass der Weißherbst ein Rosé und kein Weißwein ist, diesen nicht bestellt hätte.
Unter Kentniss der Sachlage hätte K eine solche Willenserklärung niemals abgegeben.
Somit fallen Wille und Erklärung auseinander.
Unnmitelbar nach Erhalt der Lieferung informiert K F, dass er Weißwein und keinen Rosé habe kaufen wollen.
K handelt gemäß § 242 BGB nach Treu und Glauben und fechtet laut § 119 I BGB den Kaufvertrag wegen Irrtums an.
\TODO{Gerichtsurteil + Kommentar}

\section{Anfechtungsfrist}
Gemäß § 121 BGB muss die Anfechtung unverzüglich nach Kentnissnahme des Anfechtungsgrundes erfolgen.
Dieses ist hier durch K passiert und die Anfechtungsfrist wurde eingehalten.

\section{Rücktritt nach § 346 I BGB}
Gemäß § 346 I BGB ist im Falle des Rücktritts der Gegenstand, sowie dessen gewonnener Nutzen zurückzuüberlassen.
K tritt von dem Kaufvertrag laut § 349 BGB zurück, indem er dieses gegenüber F deutlich macht.
\TODO{Gerichtsurteil + Kommentar}


\section{Neues Angebot von F an K nach §145}
Durch das neue Angebot von F an K, Weißwein in vergleichbarer Qualität zu liefern, akzeptiert F die Anfechtung von K.
Gemäß § 145 BGB ist in diesem Fall kein neuer Kaufvertrag zwischen K und F zustande gekommen, da die vorrausgesetze, direkte Annahme dieses Angebots im Sinne von § 147 durch K nicht erfolgt ist.
Das Angebot erlischt und es kommt kein neuer Vertrag zu Stande.

\section{Wirkung der Anfechtung § 142 BGB}
Willenserklärungen die für ein angefochtenes Rechtsgeschäft abgegeben wurden, werden so behandelt, als wären sie nie abgegeben wurden.
Dementsprechend ist der Vertrag zwischen K und F nach § 142 I BGB nichtig.
vertrag damit unwirksam, pflichten erloschen, etc etc...
\TODO{Gerichtsurteil + Kommentar}
