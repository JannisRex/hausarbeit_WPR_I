\chapter{Anspruch des F gegen K auf Schadensersatz nach § 122 BGB}\label{chap:122}

\section{Anspruch nach § 122 I BGB}
F könnte gegen K Anspruch auf Schadensersatz nach § 122 BGB haben.
Dazu müsste eine Schadensersatzpflicht von K gegen F vorliegen.
Eine Schadensersatzpflicht besteht, sobald eine Willenserklärung nach § 118 BGB nichtig oder auf Grund der §§ 119, 120 BGB angefochten wurde.\footnote{Vgl. \cite[§119, Rn.32]{pruett}.}\footnote{BGH, 26.1.2005, Az. VIII ZR 79/04}
K hat nach § 119 BGB den Kaufvertrag mit F erfolgreich angefochten.
Damit ein Anspruch auf Schadensersatz nach § 122 BGB geprüft werden kann, muss bestimmt werden, welche Art des Schadens ersetzt werden soll.\footnote{Vgl. \cite[§122, Rn.5]{bgb17}.}

\subsection{Negatives Interesse}
Der Schaden, welcher durch das Vertrauen auf die Gültigkeit einer Erklärung entsteht, ist negatives Interesse oder auch Vertrauensschaden.\footnote{Vgl. \cite[§122, Rn.21]{bgb20}.}\footnote{VG Gießen, 22.08.2018, Az. 6 K 6757/17.GI}
Maßgebend für den Schaden ist die Abgabe der eigenen Willenserklärung.
Es geht darum, rechtlich so gestellt zu werden, als hätte es die Willenserklärungen und folglich auch das Geschäft nicht gegeben.\footnote{Vgl. \cite[§122, Rn.23]{palandt}.}
Entschädigt wird hier die Vermögensdifferenz zwischen hypothetischen Zustand, ohne schädigendes Ereignis und dem tatsächlichen vorliegendem Zustand.

\subsection{Positives Interesse}
Positives Interesse oder auch Erfüllungsschaden beschreibt den Schaden, der dadurch entstanden ist, dass das Rechtsgeschäft nicht ordnungsgemäß durchgeführt wurde.\footnote{Vgl. \cite[§252, Rn.4]{bgb17}.}
Die Nichterbringung der Leistung ist hier ausschlaggebend für den Schadensersatz.
Folglich geht es dem Geschädigten darum, rechtlich so gestellt zu werden, als wäre das Vertrag erfüllt worden.\footnote{Vgl. \cite[§122, Rn.21]{palandt}.}\footnote{OLG Nürnberg, 5.6.2015, Az. 14 U 468/07}
Die Entschädigung beläuft sich in diesem Fall auf die Differenz des Gewinns, abzüglich der Kosten des Rechtsgeschäft.

\section{Auslegung}
Gemäß § 122 I BGB steht dem Beschädigten lediglich Ersatz auf den Schadens zu, welcher durch das Vertrauen auf die Erklärung entstanden ist.
Ein Anspruch auf den Erfüllungswert durch § 122 I BGB würde jegliche Anfechtung nach §§ 119, 120 irrelevant machen.
Demnach kann durch F nur Anspruch auf das negative Interesse erhoben werden.
Der hier vorliegende Vertrauensschaden ergibt sich durch den Verlust, welchen F durch das Vertrauen an die Willenserklärung von K erleiden musste.
Hierunter wird etwa der Transportaufwand der Lieferung und Rücknahme gezählt.
Ob der Wein beispielsweise reserviert oder von F andere Aufwände durch K entstanden sind, ist aus dem Sachverhalt nicht zu erkennen.
Die Berechnung zur Höhe des Anspruchs wird anhand Methoden wie der Differenzhypothese, oder der Vorteilsanrechnung durchgeführt.
F könnte nach § 122 I BGB Schadensersatz des Vertrauensschaden durch K fordern, sofern eine Prüfung von § 122 II BGB die Schadensersatzpflicht nicht verhindert.

\section{Anwendung von § 122 II BGB}
In § 122 II BGB wird geregelt, dass der Anspruch des Geschädigten nach § 122 I BGB entfällt, sobald der Umstand, welcher zur Anfechtung geführt hat, bekannt war oder sein sollte.\footnote{Vgl. \cite[§122, Rn.21]{bgb17}.}\footnote{BGH, 4.9.1983, Az. ui zr 100/82 }
Der Grund für den Wegfall des Anspruchs besteht darin, dass in diesen Fällen kein schutzwürdiges Vertrauensverhältnis zwischen den Vertragspartnern entstanden ist und dementsprechend keine Haftung des Erklärenden rechtfertigt.\footnote{Vgl. \cite[§15, Rn.7]{leenen}.}
Der Verkäufer F zeigte kein Interesse daran sicherzugehen, dass K sich in vollem Umfang über seine Willenserklärung bewusst ist.
Infolge von Fahrlässigkeit wusste F nicht, dass K keinen Rosé, sondern Weißwein habe kaufen wollen.
F versuchte nicht, eventuelle Missverständnisse vorab zu klären, obwohl es durch § 122 II BGB vorausgesetzt ist. 
Im Sinne von § 122 II BGB hat F die Pflicht, des Kennenmüssen der Gründe der Nichtigkeit oder der Anfechtbarkeit.
Dementsprechend muss K infolge von Fahrlässigkeit nach § 276 II BGB durch F keinen Schadensersatz gemäß § 122 I BGB leisten.
