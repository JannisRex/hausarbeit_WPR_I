\chapter{Umdeutung des Rechtsgeschäfts nach § 140 BGB}\label{chap:140}

\section{Umdeutung}
Der Kaufvertrag zwischen W und K könnte, unter Erfüllung einiger Voraussetzungen, gemäß §§ 119, 120, 140 BGB umgedeutet werden, damit im Sinne der ursprünglichen Willenserklärung, ein gültiges Rechtsgeschäft zu Stande kommen kann.
Damit eine Umdeutung erfolgen kann, muss ein nichtiges Rechtsgeschäft, genauso wie ein Anderes, welches den gleichen Erfordernissen entspricht, vorliegen.
Ein Rechtsgeschäft ist nichtig, sobald eine vollständige rechtliche Wirkungslosigkeit des Vertrags eingetreten ist.\footnote{Vgl. \cite[§140, Rn.3]{bgb17}.}
Nichtigkeit kann durch Anfechtung gemäß § 142 I BGB oder durch nicht beachten der Vorschriften aus §§ 125, 134, 138 BGB eintreten.\footnote{Vgl. \cite[§140, Rn.6]{bgb20}.}
Die Umdeutung ermöglicht folglich die Wirksamkeit eines unwirksamen Rechtsgeschäfts, wenn auch in abgeschwächter Form, aufrecht zu erhalten.\footnote{Vgl. \cite[§140, Rn.1]{faust}.}
Umdeutung nach § 140 BGB dient dazu, dem Parteiwillen, trotz nichtigem Rechtsgeschäft, soweit wie möglich zur Wirksamkeit zu verhelfen.\footnote{Vgl. \cite[§140, Rn.13]{bgb18}.}


\section{Voraussetzungen}
Grundlegend für die Umdeutung ist das Vorliegen eines Ersatzgeschäfts, für welches die Wirksamkeitsvoraussetzungen erfüllt sein müssen.\footnote{Vgl. \cite[§140, Rn.18]{bgb20}.}
Außerdem darf dieses Ersatzgeschäft in seinen Rechtsfolgen nicht umfangreicher, als das eigentliche Rechtsgeschäft sein.\footnote{Vgl. \cite[§140, Rn.20]{bgb20}.}
Genauso müssen alle Voraussetzungen für das Rechtsgeschäft vorliegen, in welches umgedeutet werden soll.
Liegt dementsprechend ein nichtiges Rechtsgeschäft, so wie ein Ersatzgeschäft vor, muss geprüft werden, ob des Rechtsgeschäft den mutmaßlichen Willen der Beteiligten vertritt.\footnote{Vgl. \cite[§140, Rn.2]{faust}.}
Demnach wird geprüft, ob durch das Ersatzgeschäft der durch das ursprüngliche Rechtsgeschäft erstrebte wirtschaftliche Erfolg in ähnlichem Umfang erreicht wird.
Entspricht das Ersatzgeschäft den mutmaßlichen Willen beider Partein, so kann eine Umdeutung nach § 140 BGB erfolgen.\footnote{BGH, 1.10.2013, Az. XI ZR 28/12}


\section{Auslegung}
F könnte das gescheiterte Rechtsgeschäft mit K in ein Ersatzgeschäft umdeuten, welches die Übereignung von Weißwein, statt Rosé beinhaltet.
Sofern eine Umdeutung gemäß § 140 BGB im Sinne von K und F geschehen kann, müssen die erörterten Voraussetzungen erfüllt werden.
Durch die akzeptierte Anfechtung gemäß §§ 119 I, 142 BGB ist die Nichtigkeit des Rechtsgeschäfts zwischen K und F erwiesen.
Umfasst der Rechtsumfang weniger oder die gleiche Anzahl an Gütern, so wie einen geringeren oder gleichen Kaufpreis, so übersteigen die Rechtsfolgen nicht die des ursprünglichen Rechtsgeschäfts.\footnote{Vgl. \cite[§140, Rn.7]{bgb19}.}
Der Umdeutung zu Grunde liegender Anfechtung von K aus §§ 119, 140 BGB geht hervor, dass der Wille von K nicht mit dem des F übereinstimmt.
Voraussetzung für eine Umdeutung nach § 140 BGB ist das Entsprechen des hypothetischen Willens beider Partein.
Durch die Anfechtung von K wird deutlich, dass er kein neues Rechtsgeschäft mit F eingehen möchte, dementsprechend ist eine Umdeutung im Sinne von § 140 BGB ausgeschlossen.
