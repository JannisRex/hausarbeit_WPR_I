\chapter{Umdeutung des Rechtsgeschäfts nach § 140 BGB}\label{chap:140}
\TODO{todo: maybe zu K gegen F}
\section{Umdeutung}
Der Kaufvertrag zwischen W und K könnte, unter Erfüllung einiger Voraussetungen, gemäß § 140 BGB umgedeutet werden, damit im Sinne der ursprünglichen Wilenserklärung, ein gültiges Rechtsgeschäft zu Stande kommen kann.
Damit eine Umdeutung erfolgen kann, muss ein nichtiges Rechtsgeschäft, genauso wie ein Anderes, welches den gleichen Erfordernissen entspricht, vorliegen.
Ein Rechtsgeschäft ist nichtig, sobald eine vollständige rechtliche Wirklungslosigkeit des Vertrags eingetreteten ist.
Nichtigkeit kann durch Anfechtung gemäß § 142 I BGB oder durch nicht beachten der Vorschriften aus §§ 125, 134, 138 BGB eintreten.
Die Umdeutung ermöglicht folglich die Wirksamkeit eines unwirksamen Rechtsgeschäfts, wenn auch in abgeschwächter Form, aufrecht zu erhalten.\footnote{Vgl. \cite[§140, Rn.1]{faust}.}

\section{Voraussetzungen}
Rechtsgeschäft muss nichtig sein.

Wille des Arbeitgebers.\footnote{Vgl. \cite[§140, Rn.2]{faust}.}