\chapter{Anspruch nach § 985}\label{chap:985}
\TODO{todo: text, stichpunkte}
\section{Anspruch entstanden}
Nicht V hat das
Rad an K übergeben, sondern dessen Angestellter A. A
ist Besitzdiener (§ 855) des V, so daß mit der Übergabe
des Rads der Besitz direkt von V auf K überging. V und
K müßten sich aber auch darüber einig gewesen sein,
daß das Eigentum auf K übergehen soll. Es ist davon
auszugehen, daß V seinen Ladenverkäufer A dazu
ermächtigt hat, im Rahmen des ordnungsgemäßen
Geschäftsablaufs Verkaufsgegenstände zu übereignen.
A war somit verfügungsberechtigt (§ 185) und vertrat V
wirksam bei der Einigung mit K (§ 164 I). Für V ist die
Einigung lediglich rechtlich vorteilhaft, so daß die
Einwilligung seines gesetzlichen Vertreters nach § 107
nicht erforderlich war.
\subsection{}