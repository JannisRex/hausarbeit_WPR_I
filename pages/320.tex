\chapter{Anspruch des W gegen K auf Gegenleistung nach § 433 II BGB}\label{chap:320}

\section{Anspruch}
W könnte gegen K den Anspruch auf Gegenleistung, in Form des vereinbarten Kaufpreis aus § 433 II BGB haben.
Sofern gemäß § 145 BGB ein Vertrag zu Stande gekommen ist, haben laut § 433 BGB beide Partein eine Pflicht zu erfüllen.\footnote{Vgl. \cite[§433, Rn.3]{palandt}.}
Zwischen K und W wurde eine verbindliche Vereinbarung getroffen, ein gültiger Vertrag liegt also vor. 
K hat demnach laut § 433 II BGB die Pflicht den vereinbarten Kaufpreis zu zahlen.

\section{Verweigerung der Gegenleistung aus § 320 I BGB}
Handelt es sich um einen gegenseitigen Vertrag, kann nach § 320 I BGB das Bewirken der Gegenleistung so lange verweigert werden, wie die Hauptleistung nicht erbracht ist.\footnote{BGH, 14.02.2020, Az. V ZR 11/18}
Durch Zurückhaltung kann die geschuldete Leistung des Anderen erfordert werden.
Die Partei, welche zur Erfüllung der Leistung bereit ist, kann sich auf § 320 I BGB berufen.\footnote{Vgl. \cite[§320, Rn.11]{bgb17}.}
K könnte die Zahlung somit bis zur Erfüllung der Pflichten von W zurückhalten.
Durch das Leistungsverweigerungsrecht aus § 275 II BGB hat W seine Pflicht gegenüber K nicht mehr zu erfüllen.
Dementsprechend muss K die Gegenleistung gemäß § 320 BGB nichtmehr erbringen.\footnote{Vgl. \cite[§275, Rn.8]{bgb18}.}

\section{Befreiung von der Gegenleistung aus § 326 BGB}
Entfällt die Pflicht der Leistung des Schuldners nach § 275 BGB, entfällt auch der Anspruch auf die Gegenleistung.\footnote{Vgl. \cite[§326, Rn.2]{bgb20}.}
Im Falle der echten Unmöglichkeit nach § 275 I BGB tritt der Wegfall der Gegenleistung automatisch ein.
Liegt eine faktische Unmöglichkeit nach § 275 II BGB vor, tritt der Wegfall ein, sobald der Schuldner die Unmöglichkeit nach § 275 II BGB geltend macht.\footnote{Vgl. \cite[§275, Rn.16]{bgb17}.}
Sofern der Gläubiger nach § 326 II BGB verantwortlich für die Unmöglichkeit ist, besteht die Gegenleistungspflicht weiterhin.\footnote{Vgl. \cite[§275, Rn.162]{mueko}.} \footnote{LG Essen, 04.02.2010, Az. 10 S 227/09}
K erfüllte seine Verantwortung und erschien zum besprochenen Termin am Weingut des W um die Ware abzuholen.
Folglich liegt die Verantwortung der Schuld nicht bei dem Gläubiger K.
W trägt die gesamte Verantwortung der Unmöglichkeit und dementsprechend findet § 326 II BGB keine Verwendung.
Die Unmöglichkeit im Sinne von § 275 II BGB wurde von W im Bezug auf das vorliegende Schuldverhältnis beansprucht. 
Die Pflicht zur Gegenleistung von K erlischt im Sinne von § 326 I BGB, auf Grund des Vorliegens einer Unmöglichkeit nach § 275 II BGB.\footnote{BGH, 03.11.2005, Az. IX ZR 140/04}
