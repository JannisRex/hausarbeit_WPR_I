\chapter{Anspruch des W gegen K auf Gegenleistung nach § 320 BGB}\label{chap:320}
\TODO{todo: Gerichtsurteil + Kommentar}
\section{Anspruch}
W könnte gegen K den Anspruch auf Gegenleistung aus § 320 BGB haben.
Sofern gemäß § 145 BGB ein Vertrag zu Stande gekommen ist, haben laut § 433 beide Partein eine Pflicht zu erfüllen. 
Durch das Leistungsverweigerungsrecht aus § 275 II BGB hat W seine Pflicht gegenüber K nichtmehr zu erfüllen.
K hat nach § 433 II die Pflicht den vereinbarten Kaufpreis zu zahlen.
\section{Befreiung von der Gegenleistung aus § 326}
Entfällt die Leistung der Pflicht des Schuldners nach § 275 BGB, entfällt auch der Anspruch auf die Gegenleistung. 
Sofern der Gläubiger nach § 326 II verantwortlich für die Unmöglichkeit ist, besteht die Gegenleistungspflicht weiterhin.\footnote{Vgl. \cite[§275, Rn.162]{mueko}.}
K erfüllte seine Verantwortung und erschien zum besprochenen Termin am Weingut des W um die Ware abzuholen.
Folglich liegt die Verantwortung der Schuld bei W.
Die Pflicht zur Gegenleistung von K erlischt, auf Grund des Vorliegens einer Unmöglichkeit nach § 275 II.
