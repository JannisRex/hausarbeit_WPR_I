\newpage
\section*{Sachverhalt}\label{sachverhalt}
K und E sind seit fast 25 Jahren verheiratet und planen für August 2020 daher eine große
Silberhochzeit. Dafür wollen sie den perfekten Wein den Gästen anbieten. Dazu fahren sie im Juni
2020 an die Mosel verkosten dort bei verschiedenen Weinbauern Weißweine. Bei dem Weingut des Winzer W finden sie dann den gewünschten Wein, ein Weißburgunder von 2017. Da W gerade viel in den Weinbergen und den Keltereien beschäftigt ist, beauftragt er seine Lebensgefährtin (L) sich um K und E zu kümmern und alles für den Verkauf klar zu machen. In diesem Sinne wird verbindlich vereinbart, das K und E 50 Flaschen des Weißburgunders im Juli 2020 bei W abholen werden. Da der L das Paar K und E sympathisch ist, gewährt sie dem Paar pro Flasche einen Sonderpreis von 10,00€ statt der sonst üblichen 12,00€.
\newline
Als K und E im Juli den Wein holen wollen, müssen sie feststellen, dass der W den Wein an den Spezialitätenhändler S verkauft und übereignet hat, der diese bereits vollständig anderweitig veräußert hat. Aus diesem Grund bleibt K und E nichts anderes übrig als beim Nachbarweingut 50 Flaschen Weißburgunder gleicher Qualität zum Paris von 14,00€ pro Flasche zu erwerben.
\newline
Kurze Zeit später nach der Silberhochzeit hat der K wiederholt Kreislaufbeschwerden. Sein Hausarzt empfiehlt ihm, es vorerst einmal täglich mit einem Glas Weißwein zu versuchen. Auf dem Nachhauseweg kommt K an der Weinhandlung des F vorbei, der im Schaufenster eine Flasche
\newline
Weißherbst zum Preis vom 3,90€ anbietet. Überrascht über das Angebot, bestellt der K gleich 100 Flaschen, ohne den Wein vorher probiert zu haben. Es wird zudem vereinbart, dass der Wein kostenlos nach Hause geliefert wird.
\newline
Bevor der Wein geliefert wird, geht der K aufgrund erneuter Kreislaufbeschwerden zu einem anderen Arzt, da sein Hausarzt im Urlaub ist. Dieser rät ihm vom Alkoholgenuss streng ab und empfiehlt lieber einen Tasse Kamillentee pro Tag zu trinken. Da seine Frau ihm das auch schon geraten hat, leuchtet dieser Rat dem K ein.
\newline
Kurz darauf wird der bestellt e Wein geliefert. Als K die erste Falsche öffnet, muss er feststellen, dass der Weißherbst gar kein Weißwein ist, sondern ein aus roten Trauben gewonnener Rosé. Solange der Wein sich in der dunkelgrünen Flasche befand, war dies nicht zu erkennen.
\newline
Dieses kommt dem K sehr gelegen. Er informiert umgehend den F, dass er Weißwein habe kaufen wollen und daher den Weißherbst nicht bezahlen werde. F erklärt sich bereit, den Weißherbst zurückzunehmen und stattdessen einen Weißwein mit vergleichbarer Qualität zu liefern. Das lehnt K ebenfalls ab, da er überhaupt keinen Wein mehr trinken werde.
\newline\\
\newline\\
\emph{Wie ist die Rechtslage?}
